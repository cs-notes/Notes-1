\ifdefined\isphone
  \documentclass[a6paper,11pt,oneside]{book}

\usepackage{algpseudocode}
\usepackage{algorithm}
\usepackage{amsfonts}
\usepackage{amsmath,amsthm,amssymb}
\usepackage{graphicx}
\usepackage{hyperref}
\usepackage{mathtools}
\usepackage{steinmetz}
\usepackage{fancyvrb}
\usepackage{textcomp}
\usepackage{gensymb}
\usepackage[normalem]{ulem}
\usepackage[T1]{fontenc}
\usepackage{cmap}
\usepackage{xspace}

\DeclareGraphicsExtensions{.pdf,.png,.jpg}

\newcommand{\mb}[1]{\ensuremath{\mathbb{{#1}}}}
\newcommand{\setof}[1]{\ensuremath{\left \{ #1 \right \}}}
\newcommand{\tuple}[1]{\ensuremath{\left \langle #1 \right \rangle }}

\DeclarePairedDelimiter{\ceil}{\lceil}{\rceil}
\DeclarePairedDelimiter{\floor}{\lfloor}{\rfloor}

\DeclareSymbolFont{bbold}{U}{bbold}{m}{n}
\DeclareSymbolFontAlphabet{\mathbbold}{bbold}

\newtheorem{statement}{Statement}
\newtheorem{rmk}{Remark}
\newtheorem{defi}{Definition}
\newtheorem{example}{Example}
\newtheorem{theorem}{Theorem}
\newtheorem{lemma}[theorem]{Lemma}
\newtheorem{proposition}[theorem]{Proposition}
\newtheorem{pf}[theorem]{Proof}
\newtheorem{corollary}[theorem]{Corollary}

% Differences
\usepackage[margin=5mm]{geometry}
\usepackage{tgheros}
\renewcommand*\familydefault{\sfdefault}

\else
  \documentclass[11pt,oneside]{book}

\usepackage{algpseudocode}
\usepackage{algorithm}
\usepackage{amsfonts}
\usepackage{amsmath,amsthm,amssymb}
\usepackage{graphicx}
\usepackage{hyperref}
\usepackage{mathtools}
\usepackage{steinmetz}
\usepackage{fancyvrb}
\usepackage{textcomp}
\usepackage{gensymb}
\usepackage[normalem]{ulem}
\usepackage[T1]{fontenc}
\usepackage{cmap}
\usepackage{xspace}

\DeclareGraphicsExtensions{.pdf,.png,.jpg}

\newcommand{\mb}[1]{\ensuremath{\mathbb{{#1}}}}
\newcommand{\setof}[1]{\ensuremath{\left \{ #1 \right \}}}
\newcommand{\tuple}[1]{\ensuremath{\left \langle #1 \right \rangle }}

\DeclarePairedDelimiter{\ceil}{\lceil}{\rceil}
\DeclarePairedDelimiter{\floor}{\lfloor}{\rfloor}

\DeclareSymbolFont{bbold}{U}{bbold}{m}{n}
\DeclareSymbolFontAlphabet{\mathbbold}{bbold}

\newtheorem{statement}{Statement}
\newtheorem{rmk}{Remark}
\newtheorem{defi}{Definition}
\newtheorem{example}{Example}
\newtheorem{theorem}{Theorem}
\newtheorem{lemma}[theorem]{Lemma}
\newtheorem{proposition}[theorem]{Proposition}
\newtheorem{pf}[theorem]{Proof}
\newtheorem{corollary}[theorem]{Corollary}

% Differences
\setlength{\textheight}{9.5in}
\setlength{\textwidth}{7.0in}
\setlength{\topmargin}{-0.75in}
\setlength{\oddsidemargin}{-0.25in}
\setlength{\evensidemargin}{0.75in}
\setlength{\parskip}{0.15in}
\setlength{\parindent}{0in}

\fi

\usepackage{listings}

\DeclarePairedDelimiter{\ceil}{\lceil}{\rceil}
\DeclarePairedDelimiter{\floor}{\lfloor}{\rfloor}

% Course Name
\title{CS 348}

% Fill in your name.
\author{Shale Craig}

% Set the default pseudocode language:
\lstset{language=SQL}

% Now we will begin writing the document.
\begin{document}
    \maketitle

    % Next, we need to make a Table of Contents, List of Figures and
    % List of Tables.  You will most likely need to run LaTeX twice to
    % get these correct.  The first pass for LaTeX to figure out the
    % labels, and the second pass to put in the right references.
    \tableofcontents

    %%%%%%%%%%%%%%%%%%%%%%%%%%%%%%%%%%%%%%%%%%%%%%%%%%%%%%%%%%%%%%%%%%%%%
    %% REPORT BODY
    %%%%%%%%%%%%%%%%%%%%%%%%%%%%%%%%%%%%%%%%%%%%%%%%%%%%%%%%%%%%%%%%%%%%%
    %% \main will make the \section commands numbered again,
    %% it will also use arabic page numbers.
    \mainmatter
    \part{Midterm Content} % (fold)
    \label{prt:midterm_ _content_}
        \chapter{Introduction} % (fold)
        \label{cha:introduction}
            \section{Motivation} % (fold)
            \label{sec:motivation}
                We're motivated to develop corporate databases in an increasingly information-oriented society.
                How do we do this? What does the fox say? Nobody knows.
            % section motivation (end)
            \section{Data Management} % (fold)
            \label{sec:data_management}
                \emph{Objective:} to represent (a part of) the world usefully while \emph{abstracting} away the details.
                \begin{description}
                    \item[Intension] of data conceptually describes the schema of a database (or data). \\
                        \verb|EMPLOYEE(SIN, NAME, ADDRESS, BIRTHDATE, SALARY)|
                    \item[Extension] of data is actually instances of data. \\
                        \verb|(1234, "Foo Bar", "123 Bar Street", Jan 1, 1970, 10$/h)|
                \end{description}
            % section data_management (end)
            \section{Components of a Database System} % (fold)
            \label{sec:components_of_a_database_system}
                \begin{itemize}
                    \item Hardware
                    \item Software
                        \begin{itemize}
                            \item Application Programs
                            \item Utility programs
                            \item OS
                        \end{itemize}
                    \item Database Management System (DBMS)

                        Dictates data structure with simple CRUD support through variety of concurrent access methods.
                \end{itemize}

                \begin{enumerate}
                    \item \emph{Concurrency Control} exists so queries are externally consistent.
                    \item \emph{Security} exists to protect data from unauthorized access (password controlled, etc).
                    \item \emph{Integrity} exists as constraints to ensure that data in the database is accurate and meaningful.
                    \item \emph{Recovery from Failures} protects the database from dying.
                \end{enumerate}
            % section components_of_a_database_system (end)
            \section{Data} % (fold)
            \label{sec:data}
                \begin{itemize}
                    \item Is the content of the knowledge of the organization.
                    \item \emph{Logical Files} as seen by application programmers - deals with the layout.
                    \item \emph{Physical Files} as seen by system programmers - performance.
                    \item Very often, the two types of files are related.
                \end{itemize}
            % section data (end)
            \section{DBMS Architecture} % (fold)
            \label{sec:dbms_architecture}
                Exists within a dynamic environment where programs are highly linked, which leads to high maintenance cost.

                We'd like to introduce independence between data and the application programs that use them, so either can be changed.

                We'd like to maintain the ability to make logical changes (to the data) explicitly, without physical changes (storage media) affecting it.

                \begin{description}
                    \item[Data Definition Language]  specifies the schemas and their mappings. (DDL)
                    \item[Data Dictionary] is the result of compilation of DDL statements in a schema.
                    \item[Data Manipulation Language] is the commands that are issued to the host program. (DML)
                    \item[Query Language] is a language for interactive data manipulation.
                \end{description}

            % section dbms_architecture (end)
        % chapter introduction (end)
        \chapter{Data Models} % (fold) 18
        \label{cha:data_models}
            Data Models are guidelines and structure for organizing value-based or object-based data and executing operations within constraints.

            \section{Entity Relationship Model (ERM)} % (fold)
            \label{sec:entity_relationship_model}
                \textbf{Entity Sets} are a thing or object that can be distinctly identified (an \textit{entity}) grouped into sets.

                This section glosses over terms such as ``entity'', ``relationship'', and ``relationship type''.

                This is the notation that goes as follows:
                \begin{description}
                    \item[Librarian] (\uline{Emp\#}, Name, Salary, Addr, Allowance, Union, LibraryName)
                \end{description}
                This indicates that librarians have the primary key \textit{Emp\#}, and the attributes specified afterwards (Name, Salary, etc).
            % section entity_relationship_model (end)
            \section{Entity-Relationship Diagram (RD)} % (fold)
            \label{sec:entity_relationship_diagram_}
                This is an extension of UML that we can quickly convert to something actually useful.

                \subsection{Structure} % (fold)
                \label{sub:structure}
                    This is a UML-like graph.
                    Since you probably don't know about the dialect of UML that this professor wants us to know, here's a quick primer of what you need to know:
                    \begin{description}
                        \item[Entity Types] are specified by rectangles.
                        These connect to Relationship Types and Attributes.
                        \item[Relationship Types] are specified by diamonds.
                        These connect $\ge$ two Entity Types.
                        \item[Attributes] are specified by ellipses.
                        These connect to single Entity Types.
                    \end{description}
                % subsection structure (end)
                \subsection{Constraints} % (fold)
                \label{sub:constraints}
                    We can establish constraints on these as follows:
                    \subsubsection{Primary Keys} % (fold)
                    \label{ssub:primary_keys}
                        There are two types of keys:
                        \begin{description}
                            \item[Candidate Keys] are minimal sets of attributes whose values identify an entity at all times.
                            \item[Primary Keys] are the main way of identifying an entity.
                            They must be a candidate key.
                        \end{description}
                    % subsubsection primary_keys (end)
                    \subsubsection{Cardinality} % (fold)
                    \label{ssub:cardinality}
                        We can force the number of entities (i.e. the cardinality of the ones involved) using UML-like $(x, y)$ notation on relationships.

                        Generally, it's expressed as $(\text{min}, \text{max})$.
                        The restrictions are $0\le x$, and $y$ can be any number or a $*$\footnote{The star indicates ``any''.}.

						The restrictions are written next to the corresponding entity.
                    % subsubsection cardinality (end)
                    \subsubsection{Existence Constraints} % (fold)
                    \label{ssub:existence_constraints}
                        By designing our ERD properly, we can create an entity $A$ which is totally dependent if every existence of $A$ is always associated with another entity through a relationship.
                    % subsubsection existence_constraints (end)
                % subsection constraints (end)
                \subsection{Extended ERM (EERM)} % (fold)
                \label{sub:extended_erm}
                    \subsubsection{Generalization and Specialization} % (fold)
                    \label{ssub:generalization_and_specialization}

                        We can extend what ERM specifies by using inheritance.

                        We can express the similar properties using the concept of \textbf{generalization}\footnote{He tends to enjoy the ``is-a'' concept more than the phrase ``generalization''.
                        I like the word ``inheritance'' more.}.

                        By adding a tree-structure, we can specify our ERDs with a tree-like structure.
                        Parent entities are connected to triangle blocks that say ``IS-A'' to their children.
                    % subsubsection generalization_and_specialization (end)
                    \subsubsection{Aggregation} % (fold)
                    \label{ssub:aggregation}
                        Supposing that we want to construct relationships to relationships\footnote{An example of this is that a \textit{Student} may \textit{Participate} in a \textit{Project}.
                        There is a relationship \textit{Eval} between the \textit{Student}, \textit{Participate}, and \textit{Project} and a \textit{Report}.
                        This is an example of this idea.}.
                        In this case, we can make a box around the elements from the first relation (almost like they're an ``entity''), and connect the relationship in the box to the external relationships desired.
                    % subsubsection aggregation (end)
                % subsection extended_erm (end)
            % section entity_relationship_diagram_ (end)
        % chapter data_models (end)
        \chapter{Storage Systems and File Structures} % (fold) 43
        \label{cha:storage_systems_and_file_structures}
            \section{Introduction} % (fold)
            \label{sec:introduction}
                Data is stored on primary or secondary storage.
                \subsection{Primary Storage} % (fold)
                \label{sub:primary_storage}
                    Primary storage tends to be faster (caches, dynamic random-access memory (DRAM)); it's fast but expensive.
                % subsection primary_storage (end)
                \subsection{Secondary Storage} % (fold)
                \label{sub:secondary_storage}
                    Secondary storage tends to be larger than large, but it's also very slow.
                % subsection secondary_storage (end)
                \subsection{Disk Storage Devices} % (fold)
                \label{sub:disk_storage_devices}
                    Disks are the most common type of secondary storage, and can hold terabytes.

                    They are constructed of \textit{disk packs} of \textit{magnetic disks} connected to a rotating spindle.
                    The disks have concentric circular \textit{tracks} on each surface.
                    Tracks with the same diameter form a \textit{cylinder}.
                    Each track is divided into equal size units called \textit{blocks} or \textit{pages}.

                    Pages are moved into main memory on demand.
                    Since the access time is ~$30$ms, and the CPU takes nano-seconds to process things, I/O is the bottleneck.
                % subsection disk_storage_devices (end)
            % section introduction (end)
            \section{Reducing Latency and Page Accesses} % (fold)
            \label{sec:reducing_latency_and_page_accesses}
                We store pages containing related information near to each other since applications are likely to read them next\footnote{Is this called the principle of locality?}.
                \subsection{Accessing Data Through a Cache} % (fold)
                \label{sub:accessing_data_through_a_cache}
                    Files are a sequence of records stored on disk blocks.

                    Records can be \textbf{fixed-length} or \textbf{variable} length, and can span only one block.

                    Physical disk blocks allocated to hold records of a file can be \textit{contiguous}, \textit{linked}, or \textit{indexed}.
                % subsection accessing_data_through_a_cache (end)
                \subsection{Ordered Files} % (fold)
                \label{sub:ordered_files}
                    Ordered files (or \textit{sequential files}) are records kept sorted by the values of an ordering field.

                    Insertion can be expensive,
					so some implementations use an \textit{overflow file} for new records to improve insertion.
					This is periodically merged with the ordered file.

                    Binary searches are used within a file to find a record with a given value of the ordering field.
                % subsection ordered_files (end)
            % section reducing_latency_and_page_accesses (end)
        % chapter storage_systems_and_file_structures (end)
        \chapter{Indexing Structures for Files} % (fold)
        \label{cha:indexing_structures_for_files}
            \section{Introduction} % (fold)
            \label{sec:introduction}
                Given an attribute value, we want to retrieve all records that match that attribute.
                Indexes are great,
				since when looking up according to an index we can get optimized lookup\footnote{We don't have optimized lookup for a non-index.}.
                This speedup is great, but it comes at cost of an index file.
            % section introduction (end)
            \section{Types of Indexes} % (fold)
            \label{sec:types_of_indices}
                There are a few types of indexes:
                \begin{description}
                    \item[Search Key] is the set of attributes on which an index is built and not the one that's always built.
                    \item[Primary Index] is the index that the index for a set of entities is built around.
                    Usually, this is the search key\footnote{In the cases where the primary index isn't the search key, the search key is known as the secondary index.}.
                    \item[Ordered] is a term referring to the search key ordering in the index file.
                    Often they are ordered, but if they're not the primary index, they are \textit{unordered}.
                \end{description}
            % section types_of_indices (end)
            \section{B$^+$ Trees} % (fold)
            \label{sec:b_trees}
                B$^+$ trees are dynamic index-based data structures that are made up index and data blocks.
                They are a special type of tree optimized to reduce the number of page misses.
                \subsection{Specification} % (fold)
                \label{sub:specification}
                    For a B$^+$ tree of order $m$ and a maximum data node size of $d$\footnote{Often, $m \ne d$.}, we work under the constraints:
                    \begin{itemize}
                        \item All values are in leaf (``data'') nodes.
                        \item All leaves are on the same level.
                        \item With the exception of the root, every node has $[\floor{\frac{m-1}{2}}, m-1]$ keys, which are sorted in ascending order.
                        \item An internal (``index'') node with $k$ keys has $k+1$ pointers to children on the next level\footnote{The children correspond to the partition induced on the key space by the $k$ keys.}.
                        \item Data nodes have $[\floor{\frac{d}{2}}, d]$ sorted records, and a pointer to the next and previous data node.
                    \end{itemize}
                % subsection specification (end)
				Modifying this tree runs in logarithmic time.

				The data nodes of a B$^+$ tree contain pointers to the next and previous data node for efficient iteration.
				The code to update this the next and previous data node would be found in the functions \emph{split} and \emph{merge}.

                \subsection{Insertion} % (fold)
                \label{sub:insertion}


                    Insertion can be expressed as the following pseudocode:
                    \begin{verbatim}
    insert(node n, btree b):
        node = b.findLeaf(n)
        while (true):
            node.insert(n)
            if (!node.isOverflow())
                return
            if (node.isRoot()):
                split(node)
                return
            n = split(node)
            node = n.getParent;
                    \end{verbatim}
                    Basically, we insert into the node.
                    If we don't have enough room inside the data node, we split it (and insert the split nodes into the parent structure).

                    \subsubsection{Split - Data Node} % (fold)
                    \label{ssub:split_data_node}
                        In the case that we call split on a data (leaf) node $n_j$, we create a new node $n_{j+1}$ that will contain half the records of the old root node.
                        For an odd number of records, keep the extra record in node $n_j$.
                        Promote\footnote{Just clone the value, don't remove it from the data node.} the largest key value to the parent index node.
                    % subsubsection split_data_node (end)
                    \subsubsection{Split - Index Node} % (fold)
                    \label{ssub:split_index_node}
					In the case that we call split on a index node with keys $k_j \to k_{j+n}$, we partition the values into $[k_j \cdots k_{j+ \lfloor\frac{n}{2}\rfloor - 1}]$ and $[k_{j+ \lfloor\frac{n}{2}\rfloor + 1} \cdots  k_{j+n}]$.
                        For an odd number of records, keep the extra record in node $n_j$.
                        Move\footnote{Move, not clone.} the key $k_j$ to the parent node, and keep pointers to the left and right nodes in the data structure.
                    % subsubsection split_index_node (end)
                % subsection insertion (end)
                \subsection{Deletion} % (fold)
                \label{sub:deletion}
                    Deletion seems like a pretty annoying algorithm because leaf height needs to be maintained, but it's surprisingly simple:
                    \begin{verbatim}
    def delete(index idx, bTree b):
        node toDelete = b.find(idx)
        entry = toDelete.getEntry()
        while (entry != null):
            node parent = entry.parent
            parent.remove(entry)
            if (!parent.isUnderFlow()):
                return
            if (parent.isRoot()):
                collapseRoot()
            if (!parent.leftNeighbor.isMinimal()):
                redistribute(parent, parent.leftNeighbor)
            merge(parent, parent.leftNeighbor);
            entry = parent.getEntry()
                    \end{verbatim}
                    \subsubsection{Redistribute - Data} % (fold)
                    \label{ssub:redistribute_data}
                        Redistribute records as evenly as possible between two siblings, updating the parent key accordingly.
                    % subsubsection redistribute_data (end)
                    \subsubsection{Redistribute - Index} % (fold)
                    \label{ssub:redistribute_index}
                        Redistribute indexes as evenly as possible between two siblings, updating the parent key accordingly.
                    % subsubsection redistribute_index (end)
                    \subsubsection{Merge - Data} % (fold)
                    \label{ssub:merge_data}
                        Merge the two data blocks into one (move the records from right to left), and delete the parent entry that divided the two.
                    % subsubsection merge_data (end)
                    \subsubsection{Merge - Index} % (fold)
                    \label{ssub:merge_index}
                        One sibling has below minimum, while the other has minimum.
                        Merge the two, and delete the parent key separating the two.
                    % subsubsection merge_index (end)
                % subsection deletion (end)
            % section b_trees (end)
        % chapter indexing_structures_for_files (end)
        \chapter{Relational Model} % (fold) 76
        \label{cha:relational_model}
            Relational models are used to design and model relational databases, foo'!

            \section{Terminology} % (fold)
            \label{sec:terminology}
                We only have one data structuring tool - a \textbf{relation}.
                We can express relations as $D_1 \times \cdots \times D_n = \{ \langle a_1, \cdots, a_n \rangle | \forall i: a_i \in D_i\}$.
                We call $r = \langle a_1, \cdots a_n\rangle$ a relation on $n$-sets;
                i.e. $r$ is a set of \textbf{tuples} (often called \textbf{rows}).
                We call $D_j$ the $j^{\text{th}}$ domain of $r$, where $r$ is of degree $n$.

                When writing databases, we use the \textbf{Closed World Assumption} to govern what exists - the assumption that everything not currently know to be true is false\footnote{i.e. if I don't know about it, it must not exist.}.

                \textbf{Relational schemes} define the composition (intension) of a relation.

                \textbf{Domains} are the limited scope that an attribute is valid under.
            % section terminology (end)
            \section{Key Propagation} % (fold)
            \label{sec:key_propogation}
                For many:many relationships (i.e. many \verb|doctor|s may be the \verb|attendingDoctor| for a many \verb|patient|s), it's a good idea to keep keys joining the two in a separate relation.
            % section key_propogation (end)
            \section{Constraints} % (fold)
            \label{sec:constraints}
                \subsection{Inherent Constraints} % (fold)
                \label{sub:inherent_constraints}
                    Inside a relationship, there are a few basic constraints:
                    \begin{itemize}
                        \item We need to have a \textit{Candidate Key} that is a minimal set of attributes to uniquely identify tuples.
                        This candidate key cannot be repeated.
					\item We need to have a \textit{Primary Key} that cannot be updated, duplicated,
						nor contain nulls (\textbf{entity integrity rule})
                    \end{itemize}
                % subsection inherent_constraints (end)
                \subsubsection{Domain Constraints} % (fold)
                \label{ssub:domain_constraints}
                    In core SQL-99, we can limit attribute values using the \verb|check| constraint or creating a new domain.
                    The domain is used in schema declarations, and is a schema element:
                    \begin{verbatim}
    CREATE DOMAIN Grades CHAR(1)
        CHECK (VALUE IN ('A', 'B', 'C', 'D', 'F'));
                    \end{verbatim}
                    We can use this ``type'' later:
                    \begin{verbatim}
    create table Transcript (
        ...,
        grade Grades,
        ...
    )
                    \end{verbatim}
                % subsubsection domain_constraints (end)
                \subsubsection{Foreign Key (Referential Integrity) Constraints} % (fold)
                \label{ssub:foreign_key_referential_integrity_constraints}
                    Given that a set of attributes $FK \subseteq R_1$ is a \textit{foreign key} that references $R_2$, we have two constraints to satisfy:
                    \begin{enumerate}
                        \item The attributes of $FK$ are defined on the same domains as the primary key $PK$ of $R_2$.
                        \item A value of $FK$ either occurs as a value of $PK$, or $FK$ is null.
                    \end{enumerate}
                    We say that $R_1$ references $R_2$\footnote{He took extra time to point out that $R_1$ is the \textit{referencing} relation, and that $R_2$ is the \textit{referenced} relation with respect to this foreign key constraint.}
                % subsubsection foreign_key_referential_integrity_constraints (end)
            % section constraints (end)
            \section{Tables} % (fold)
            \label{sec:tables}
                We use tables as our basic type of represent relations.

                The SQL syntax is as follows:
                \begin{verbatim}
    create table <tname> (
        (columnDec)+
        [, (tableConstraint)]
    );
    columnDec =
        <colName> <colDataType> [default <value>][(colConstraint)+];
    colConstraint = {
        not null |
        [constraint <name>] unique |
        primary key |
        check (search_cond) |
        references <tname> [<colName>] [(refEffect)]
    }
    tableConstraint = [constraint name] {
        UNIQUE (<colName>+) |
        foreign key (<colName>+) references <tname> [(refEffect)]
    }
    refEffect = on {update | delete} (effect)
    effect = {
        set null |
        no action (restrict) |
        cascade |
        set default
    }
                \end{verbatim}
            % section tables (end)
        % chapter relational_model (end)
    % part midterm_ _content_ (end)
\end{document}
